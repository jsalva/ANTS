\documentclass[11pt]{article}
\newcommand{\X}{{\bf X}}
\newcommand{\x}{{\bf x}}
\newcommand{\y}{{\bf y}}
\renewcommand{\v}{{\bf v}}
\renewcommand{\textfraction}{0.05}
\renewcommand{\topfraction}{0.95}
\renewcommand{\bottomfraction}{0.95}

\usepackage{geometry}                % See geometry.pdf to learn the layout options. There are lots.
\usepackage{verbatim}
\usepackage[usenames]{color}
\usepackage{listings}
\lstset{breaklines=true,basicstyle=\ttfamily,breakindent=0pt}
\geometry{letterpaper}                   % ... or a4paper or a5paper or ... 
%\geometry{landscape}                % Activate for for rotated page geometry
%\usepackage[parfill]{parskip}    % Activate to begin paragraphs with an empty line rather than an indent
\usepackage{graphicx}
\usepackage{amssymb}
\usepackage{epstopdf}
\DeclareGraphicsRule{.tif}{png}{.png}{`convert #1 `dirname #1`/`basename #1 .tif`.png}

\begin{document}
%\maketitle
\noindent
David Kennedy, Ph.D.,\\
Co-Editor-in-Chief, Neuroinformatics\\

\noindent
Dear Dr. Kennedy and Reviewers,\\

\noindent
My coauthors and I would like to thank the editors and reviewers for
handling our manuscript entitled ``An Open Source Multivariate Framework for $n$-Tissue
  Segmentation with Evaluation on Public Data.''  The reviews were extremely helpful in highlighting both
the strengths and weaknesses of the previous version of the paper.  In
summary, the review found the work to be worthy of publication, with
revisions that focused on improving clarity of some key points. We make every effort, in this response letter and the
associated revision, to clarify our work and improve the paper
according to the reviewers' suggestions. We believe that we carefully
revised the manuscript to address all concerns and provide responses
to the reviewers' comments, which you will find below.   We
also elected to add detail regarding the ICM strategy
available in Atropos. Thus, we would
like to resubmit the revised manuscript which we hope will be suitable
for publication. \\

\noindent
The revision of the paper includes:
\begin{enumerate}
\item  A new figure revision highlighting ICM and providing a simple
  starter example for Atropos usage. 
\item Fixes to two typos in the equations. 
\item  Many minor changes in the body of the text.  
\item  The references were altered according to reviewer suggestions.
\end{enumerate}
We express our sincere gratitude to
the editorial staff and to the reviewers for their efforts in
evaluating this paper. \\

\noindent
Sincerely,\\

\noindent
Brian Avants, Ph.D.\\
Penn Image Computing and Science Laboratory \\ 
University of Pennsylvania\\
3600 Market Street, Ste 370\\
Philadelphia, PA 19104
\newpage
\begin{center}
\LARGE{{\bf Response To Reviewers}}
\end{center}
\section{Reviewer 1}
We thank reviewer 1 for his/her careful reading for technical content
and for improving our work.\\
\newline
\newline
{\em 1) Page 5, 2nd line from bottom. What are silver standard
  labellings?}
\newline
\newline
We've changed this line to be more specific in the current revision.
\newline
\newline
Done.  
\newline
\newline
{\em 2) Page 10, Equations 8 - 11. These follow standard
  approaches. However, Equation 11 does not seem consistent with [8]
  and [10]. If delta(x\_i,x\_j)=1 for x\_i=x\_j, then U(x) as defined in
  [10] increases, and p(x) as defined in [8] decreases - hence
  labelling similar regions seems to be penalised. However, things
  seem consistent if equation 11 is changed as in the standard graph
  cuts approaches - e.g. see equation 3 Boykov \& Jolly ICCV 2001.}
\newline
\newline
We appreciate the reviewer catching this error in the equation.  If $x_i = x_j$ then $V=0$ and is
1 otherwise, as is standard in Besag's approach. 
\newline
\newline
{\em 3) Page 10, Equation 11. Computing the value of U(x) only
  considers if neighbours are the same or not. However, as this paper
  deals with segmenting a large number of labels why was an approach
  such as that taken in Noe \& Gee IPMI 2001 not considered -
  i.e. learning the probability of labels being neighbours and using
  these?}
\newline
\newline
This is an interesting point and we intend to investigate this
direction of research in future work.  In fact, some basic
functionality along these lines is available within the guts of the
code but additional comparative research into the myriad partial volume
estimation models that have been proposed is required to determine the
best approach.  Thus, we added comments in the discussion to indicate
our interest in pursuing partial volume models but elect not to pursue
this task for this resubmission. 
\newline
\newline
{\em 4) Page 12, lines 10-14. Authors comment that applying graph cuts
  to more than one label is not globally optimal and that EM is a
  reasonable and efficient alternative. Although this is true,another
  property of graph cuts is that it is fast. How does the EM approach
  compare in terms of speed?}
\newline
\newline
Fast is a relative term and we cannot comment on the speed of a method
for which we do not have an optimal implementation.  
The main computational challenge with graph cuts is memory.  Because one needs to
solve multiple shortest path problems---and store the solution of
each---the potential for memory bloat is high.  Even in EM, memory
concerns emerge when enough labels are used (e.g. in our examples).
In theory, though, EM computations are embarassingly parallel which is
not a property shared by graph-based methods.  As such, one would
expect even an optimally implemented multi-label graph cut method to
lag in speed behind an optimally implemented EM segmentation. 
\newline
\newline
{\em 5) Page 13, Equation 17. This looks like a multivariate equation to me. I'm also not familiar with the form of Equation 19 - why is it different from 17?'}
\newline
\newline
The difference between equations 17 and 19 is that the input to the
data term is a vector (rather than scalar) and the scalar variance
parameter is replaced by the covariance matrix.  This is a standard
extension from univariate to multivariate probability. 
\newline
\newline
{\em 6) Page 17. Penultimate line. "Three likelihood classes have been
  developed, one parametric and one ...". Should this be "Three
  likelihood classes have been developed, one parametric and [two]
  ..."?'}
\newline
\newline
Yes, thank you.  The fix is made.  
\newline
\newline
{\em 7) Page 26 Figure 5. In the sign column the signs for the Left
  and Right Caudate Nucleus and Thalamus should be -ve as according to
  the data in the table, Atropos performed significantly worse than
  the majority vote.}
\newline
\newline
Fixed.  
\newline
\newline
{\em 8) I could not verify the validity of all the references as they were not numbered neither were they in alphabetical order. }
\newline
\newline
{\bf FIXME NICK Standard neuroinformatics style?}
\newline
\newline
The reviewer also mentioned not being able to locate the 19 Hammers
datasets.  It is unfortunate that the brain-development website is not
well annotated.  We suspect the reviewer looked at the IXI data which
is not the Hammers data.  We suggest the reviewer look at the adult
atlas section of the website which, currently, contains the data we
used.  This is noted in the revised text. 
\newline
\newline
We also followed each of the reviewer's suggestions on the structure
and grammar of the text. 
\newline
\newline
We appreciate reviewer 1's time and effort.  
\newline
\newline
\section{Reviewer 2}
We thank reviewer 2 for his/her attention to the manuscript and overall 
positive comments.  We made the reviewer's suggested minor changes in this revision.  
The following comments were addressed: 
\newline
\newline
{\em I was very pleased with the clarity of most of the article,
 particularly with the careful introductory explanations,
 demonstrations of the command-line arguments for Atropos, and the
 excellent review of the literature. I was also happy to see the
 multivariate results in Figure 2, and would have liked to have seen
 accompanying quantitative results. If such results were to assert the
 significant improvement possible when including a complementary
 modality (i.e., without bias), that would be a very nice contribution
 in itself.}
\newline
\newline
We agree with the reviewer that establishing the benefit of
multivariate segmentation in the presence of bias would be
significant.  However, because we only have one multivariate dataset
on which to test this hypothesis, it is difficult to do much more than
provide suggestive results, as we've done here.
\newline
\newline
{\em I have only two concerns with the article. First, there is almost
  no explanation of any of the figures in the body of the text, which
  raises questions about the significance of the findings, such as
  those of Figure 4 ("The results show that the PriorProbabilityMaps
  with w = 0.5 (far right) gives the best performance for all
  tissues."). 
\newline
\newline
The revision adds explanatory material in the body of the text
associated with each figure.  {\bf FIXME}
\newline
\newline
{\em Second, why is it that "In this study, we labeled the brain web (sic) template very sparsely, quickly and crudely and did not expect highly accurate results."
The results depending on these manual prior labels aren't informative
-- they don't give good results, and don't tell us whether the results
would be better if better manual labels were used.}
\newline
\newline
We agree with the reviewer and removed this comment and the evaluation
results.  We added: {\color{red}{We provide this capability to allow the user to implement
    an interactive editing and segmentation loop.  The user may
    run Atropos with sparse manual label guidance, evaluate the
    results, update the manual labels and repeat until achieving the desired
    outcome.  This processing loop be performed easily with, e.g., ITK-SNAP.}
\newline
\newline
{\em On page 22: "The advancements introduced with N4 permit such an
  adaptive integration with Atropos." What relevant advances are
  introduced with respect to N3?}
\newline
\newline
{\bf FIXME NICK}
\newline
\newline
{\em Equations: Please remove punctuation following equations, because it came sometimes be distracting. For example, on pg.9, eq.1's comma might also appear to be a prime/apostrophe for the variable y.}
\newline
\newline
Done except in cases where the equation ends a sentence.   
\newline
\newline
Again, we thank each of the three reviewers for their efforts.  We
truly feel that their reading of the paper was excellent and greatly
improved the work.

% \bibliographystyle{elsarticle-harv.bst}
% \bibliography{ftdcbdlong,avantsAll} 


\end{document}  



